\documentclass[]{article}
\usepackage{lmodern}
\usepackage{amssymb,amsmath}
\usepackage{ifxetex,ifluatex}
\usepackage{fixltx2e} % provides \textsubscript
\ifnum 0\ifxetex 1\fi\ifluatex 1\fi=0 % if pdftex
  \usepackage[T1]{fontenc}
  \usepackage[utf8]{inputenc}
\else % if luatex or xelatex
  \ifxetex
    \usepackage{mathspec}
  \else
    \usepackage{fontspec}
  \fi
  \defaultfontfeatures{Ligatures=TeX,Scale=MatchLowercase}
\fi
% use upquote if available, for straight quotes in verbatim environments
\IfFileExists{upquote.sty}{\usepackage{upquote}}{}
% use microtype if available
\IfFileExists{microtype.sty}{%
\usepackage{microtype}
\UseMicrotypeSet[protrusion]{basicmath} % disable protrusion for tt fonts
}{}
\usepackage[margin=1in]{geometry}
\usepackage{hyperref}
\hypersetup{unicode=true,
            pdftitle={How to: Het maken van een blogpost},
            pdfauthor={Diego Staphorst},
            pdfborder={0 0 0},
            breaklinks=true}
\urlstyle{same}  % don't use monospace font for urls
\usepackage{graphicx}
% grffile has become a legacy package: https://ctan.org/pkg/grffile
\IfFileExists{grffile.sty}{%
\usepackage{grffile}
}{}
\makeatletter
\def\maxwidth{\ifdim\Gin@nat@width>\linewidth\linewidth\else\Gin@nat@width\fi}
\def\maxheight{\ifdim\Gin@nat@height>\textheight\textheight\else\Gin@nat@height\fi}
\makeatother
% Scale images if necessary, so that they will not overflow the page
% margins by default, and it is still possible to overwrite the defaults
% using explicit options in \includegraphics[width, height, ...]{}
\setkeys{Gin}{width=\maxwidth,height=\maxheight,keepaspectratio}
\IfFileExists{parskip.sty}{%
\usepackage{parskip}
}{% else
\setlength{\parindent}{0pt}
\setlength{\parskip}{6pt plus 2pt minus 1pt}
}
\setlength{\emergencystretch}{3em}  % prevent overfull lines
\providecommand{\tightlist}{%
  \setlength{\itemsep}{0pt}\setlength{\parskip}{0pt}}
\setcounter{secnumdepth}{0}
% Redefines (sub)paragraphs to behave more like sections
\ifx\paragraph\undefined\else
\let\oldparagraph\paragraph
\renewcommand{\paragraph}[1]{\oldparagraph{#1}\mbox{}}
\fi
\ifx\subparagraph\undefined\else
\let\oldsubparagraph\subparagraph
\renewcommand{\subparagraph}[1]{\oldsubparagraph{#1}\mbox{}}
\fi

%%% Use protect on footnotes to avoid problems with footnotes in titles
\let\rmarkdownfootnote\footnote%
\def\footnote{\protect\rmarkdownfootnote}

%%% Change title format to be more compact
\usepackage{titling}

% Create subtitle command for use in maketitle
\providecommand{\subtitle}[1]{
  \posttitle{
    \begin{center}\large#1\end{center}
    }
}

\setlength{\droptitle}{-2em}

  \title{How to: Het maken van een blogpost}
    \pretitle{\vspace{\droptitle}\centering\huge}
  \posttitle{\par}
    \author{Diego Staphorst}
    \preauthor{\centering\large\emph}
  \postauthor{\par}
      \predate{\centering\large\emph}
  \postdate{\par}
    \date{2019-12-06}


\begin{document}
\maketitle

\hypertarget{introductie}{%
\section{Introductie}\label{introductie}}

De voornaamste intentie van deze website is het delen van onze kennis en
het open source houden van onze kennis. Dit willen wij bereiken door
tech reports te schrijven, hiermee gaan wij onze processen en resultaten
van projecten beschrijven. Door het creeeren van deze post willen wij
anderen (werknemers \& studenten) de mogelijkheid bieden om te kunnen
bijdragen aan deze website, door in eerste instantie tech reports te
maken over nieuwe tools en/of technieken gebruikt voor het oplossen van
verschillende vragen.

\hypertarget{wat-voor-tools-worden-gebruikt}{%
\subsection{Wat voor tools worden
gebruikt?}\label{wat-voor-tools-worden-gebruikt}}

Doordat er voornamelijk in R wordt gewerkt is het handig om gebruik te
maken van RStudio, omdat het gemaakt is om de gebruiker productiever te
maken. Je zou ook gebruik kunnen maken van je favoriete code editor,
maar dit wordt afgeraden.

\href{https://rstudio.com/products/rstudio/download/}{Installeer RStudio
hier}, volg de stappen op de website!

\hypertarget{hoe-kun-je-bijdragen}{%
\section{Hoe kun je bijdragen}\label{hoe-kun-je-bijdragen}}

Er zijn verschillende manieren om te helpen aan de website, in deze
blogpost focussen we ons op het maken van blogposts. Een voorwaarde is
dat je een github account hebt, een basisbegip van git en github hebt.
Hieronder staan een aantal resources, om je kennis qua git en github op
te frissen!

\begin{itemize}
\tightlist
\item
  \href{https://github.com}{Github.com}
\item
  \href{https://guides.github.com/}{Github guides}
\end{itemize}

Git kan op verschillende manieren worden gebruikt, zo kan je dat doen
via de terminal, maar ook via een Graphical User Interface (GUI). Zelf
gebruik ik \href{https://www.gitkraken.com/}{GitKraken}, het geeft een
duidelijk overzicht wat er gebeurt in de repository. Zij bieden ook
trainingen aan die de flow van github intuitiever maken,
\href{https://www.gitkraken.com/learn-git}{Learn Git with Gitkraken}.

Nu zou je je eerste bijdragen moeten kunnen maken aan dit project! Laten
we beginnen hiermee, door deze
\href{https://www.youtube.com/watch?v=j_qpzND5yAg}{video} te bekijken.
Een samenvatting van de stappen staan hieronder:

\begin{enumerate}
\def\labelenumi{\arabic{enumi}.}
\tightlist
\item
  Vork de repository en kloon hem naar je desktop
\item
  Maak een branch voor je blogpost
\item
  Pas de bestanden aan/Maak je post (hoe dit moet staat in het kopje
  hieronder)
\item
  Commit naar veranderingen naar je gemaakte branch
\item
  Maak een nieuwe pull request
\item
  Wacht op review van je pull request
\end{enumerate}

Deze stappen staan uitvoerig beschreven in de volgende link
\href{https://akrabat.com/the-beginners-guide-to-contributing-to-a-github-project/}{beginners
guide to contributing}.

\hypertarget{je-eerste-post-maken}{%
\subsection{Je eerste post maken}\label{je-eerste-post-maken}}

Als stap 1 en 2 hebt uitgevoerd op je eigen desktop, gaan we beginnen
met het maken van een blogpost.

\begin{enumerate}
\def\labelenumi{\arabic{enumi}.}
\tightlist
\item
  Open het R project in
\end{enumerate}

\href{report.pdf}{Download tech report}


\end{document}
